\documentclass{llncs}

\usepackage[utf8]{inputenc}
\usepackage{amsmath}
\usepackage{hyperref}
\usepackage[authoryear]{natbib}
\usepackage{xcolor}

% working title, please change
\title{Python as a machine learning language:
       experiences from the \textit{scikit-learn} project}

\author{Mathieu Blondel\inst{1} \and
        Lars Buitinck\inst{2} \and
        Gilles Louppe\inst{3} \and
        Jake Vanderplas\inst{4}}

\institute{Graduate School of System Informatics, Kobe University \and
           Informatics Institute, University of Amsterdam \and
           Department of EE \& CS, University of Liège \and
           Astronomy Department, University of Washington}

% Dutch name sorting hack as per http://tex.stackexchange.com/a/40750/2806
\DeclareRobustCommand{\VAN}[3]{#2}

\begin{document}

\maketitle

\begin{abstract}
The \textit{scikit-learn} project is an increasingly popular open source machine
learning library designed to be simple and efficient, accessible to non-experts,
and reusable in various contexts. In this paper, we present and discuss our
design choices for the application programming interface (API) of the project.
In particular, we describe the simple and limited interface shared by all
learning and processing units in the library and then discuss its advantages in
terms of composition and reusability. The paper also comments on implementation
details specific to the Python ecosystem and highlights obstacles faced by users
and developers of the library.
\end{abstract}

\textcolor{red}{TODO: focus on the experience we accumulated (general remark).}

\section{The \textit{scikit-learn} project}

The \textit{scikit-learn} project~\citep{pedregosa2011} is an open source
machine learning library for the Python programming language. The ambition of
the project is to provide  effective and long-established machine learning tools
within a programming environment that is accessible to non-machine learning
experts, efficient, and reusable in various scientific areas. In itself, the
project is a library of machine learning tools  rather than  a novel domain-
specific programming language. Among other things, the library includes
classical learning algorithms, model selection tools and preprocessing
procedures. From its core, the library has been designed to tie in with the set
of numeric and scientific packages centered around the Numpy and
SciPy~\citep{vanderwalt2011, varoquaux2013scipy} libraries. As such, \textit
{scikit-learn} is intended to be used within Python scripts and requires some
Python programming knowledge. At the time, it offers no command line or
graphical user interface.

Started in 2009, \textit{scikit-learn} counts a large team of volunteer
international developers mainly consisting of scientists and researchers from
various fields (e.g., machine learning, natural language processing or
neuroscience). The project also greatly benefits from many occasional
contributors proposing small code and/or documentation fixes or improvements.
Because of this large base of developers, important efforts are deployed to make
the project maintainable. In particular, we emphasize on high-quality and well-documented
code  in order to ease future developments by external contributors.
\textcolor{red}{TODO: Mention some good coding practices (automatic testing, code
reviews, etc)}

Within the past few years, the project has proven to be valuable to an  ever
increasing user base. Its popularity can be gauged from various indicators,
including hundreds of citations in scientific papers, successes in various
machine learning challenges, or significant statistics\footnote{\textcolor{red}{Cite some
statistics here.}} derived from our repositories and mailing lists.

\textcolor{red}{TODO: The rest of this text is organized as follows: in Section 2, ...}


% % XXX the following is more of a general introduction than a project vision
% scikit-learn \citep{pedregosa2011} is an open source machine learning library
% for the Python programming language,
% intended to tie in with the set of numeric and scientific packages
% centered around the NumPy and SciPy libraries
% \Citep{vanderwalt2011, varoquaux2013scipy}.
% It is a library, rather than a novel (domain-specific) programming language,
% intended to be called from Python programs;
% at present, it offers no command line interface or GUI.

% The scikit-learn project started in 2009 and has since been developed
% by a team largely consisting of scientists
% (both from computer science and other fields).
% Its popularity can be gauged from various sources:
% % TODO: list volume on the mailing list, number of contributors,
% % number of issues opened and closed, number of forks, number of surveys filled in.
% at the time of writing,
% the popular programmer's Q~\&{}A site StackOverflow\footnote{
%   \url{http://stackoverflow.com/questions/tagged/scikit-learn}}
% % update the following figure when paper is finalized
% lists 322 scikit-learn-related questions.
% The version control logs for the development version of scikit-learn
% % ddaa494c116e3c16bf032003c5cccbed851733d2
% show 183 contributors to the project's source code and documentation
% (after manual deduplication).

% The main goal of the scikit-learn project
% is to provide high-quality implementations
% of ``classic'' machine learning algorithms
% with a consistent API.

\section{API}

In this section, we discuss the general principles of the application
programming interface (API) of \textit{scikit-learn}. We first specify the
representation of data and then proceed by describing the interface shared by
all objects of the library. We then discuss the composition and reusability
advantages of that interface.

\subsection{Data representation}

In most machine learning tasks, data is modeled as a set of variables.  For
example, in a classification or regression task, the goal is to find a mapping
between input variables $X_1, ..., X_p$ -- called attributes or features -- and
some output variable $Y$. A sample is then defined as a tuple of values $(x_1,
..., x_p, y)$ of these variables and a dataset as a collection of such samples.
As such, matrices of numerical values\footnote{In the case of categorical
variables, numerical values can be obtained using various encoding schemes, such
as one-hot encoding.} constitute a natural representation used almost
unanimously within all areas of machine learning. In particular, each row in the
matrix then corresponds to a sample of the dataset and each column to one of the
variables of the problem.

In \textit{scikit-learn}, we chose a representation of data that is as close as
possible to the matrix representation: datasets are encoded as NumPy
multidimensional arrays for dense data and as SciPy sparse matrices for sparse
data. While those may seem like bare and minimalistic data representations
when compared to more object-oriented
representations\footnote{Weka~\citep{hall2009weka} is an example of a toolkit-specific
object-oriented data representation.}, they bring the prime advantage
of allowing us to rely on all NumPy and SciPy vector operations -- often orders
of magnitude faster that the corresponding Python loops -- while allowing us at
the same time to keep the code short and readable. From a practical point of
view, these formats also provide a bunch of data loading and conversion tools
which make them very easy to use in many contexts\footnote{For tasks where the
inputs are text files or semi-structured objects, we provide \textit{vectorizer}
objects that efficiently convert such data to the NumPy or SciPy formats.}.
Finally, this design choice has also been motivated by the fact that, given
their pervasiveness in many other scientific Python packages, many scientific
users of Python are likely to be already familiar with those formats, hence helping
them to familiarize with the project.


% For a machine learning library, one of the most important decisions we had to
% make is how to represent data.  Rather than reinventing the wheel, we opted for
% NumPy multidimensional arrays \Citep{vanderwalt2011} for dense data and SciPy
% sparse matrices for sparse data.  While those may seem like a ``bare''
% representation of data when compared to more object-oriented representations
% (Weka is an example of this style of representation, \citealp{hall2009weka}),
% it brings the prime advantage of allowing us to rely on NumPy and SciPy's
% \textit{vector operations} which are often orders of magnitude faster
% than the corresponding Python loops,
% while at the same time keeping the code short and readable.
% Conversion to NumPy and SciPy formats is usually easy and many scientific users
% of Python will already be used to such formats, since they are pervasive in
% other scientific Python packages.  For tasks where input is likely to consist
% of text files or semi-structured objects, we provide ``vectorizers'' -- objects
% that efficiently convert such data to the NumPy or SciPy formats.

% FIXME: elaborate what is meant below
% Lars: elaborated, but I'm not sure how important this is.
% It mostly reflects experiences with OO libraries that require conversions
% between types just to pass information to a method.
%Finally, offering a custom datatype for samples would require a conversion
%\textit{in any case}, since such a datatype would be toolkit-specific
%and no program would store its data using such a type
%if it weren't from the outset designed to use scikit-learn.

% In the remainder of the paper, we use the following notation. Training data are
% denoted by \Xtr ~and test data (i.e., unseen data to which the learning
% algorithm needs to be able to generalize) by \Xte.  Following the discussion above,
% we represent \Xtr ~and \Xte ~by 2-dimensional NumPy arrays or SciPy sparse
% matrices. Using NumPy/SciPy's notation, \Xtr\texttt{[i, j]} denotes the
% $i^\textrm{th}$ training sample's $j^\textrm{th}$ feature. For supervised learning tasks,
% such as classification or regression, we denote the training and test labels by
% \ytr ~and \yte, respectively. We typically store \ytr ~and \yte ~as
% 1-dimensional NumPy arrays (\ytr\texttt{[i]} denotes the $i^\textrm{th}$ training
% sample's label value).

\subsection{Estimators, predictors and transformers}

All objects within \textit{scikit-learn} share a uniform common basic API
consisting of three complementary interfaces: an \textit{estimator} interface
for building and fitting objects, a \textit{predictor} interface for making
predictions and a \textit{transformer} interface for converting data.
As much as possible, our design choices have been guided so as to avoid
the proliferation of framework code. We try to adopt simple conventions and
to limit to a minimum the number of methods an object must implement.

\textbf{Estimators.} The \textit{estimator} interface is at the core of the
library. It defines instantiation mechanisms of objects and exposes a
\texttt{fit} method for learning a model from training data.  All supervised and
unsupervised learning algorithms (e.g., for classification, regression or
clustering) are offered as objects implementing that interface. Machine
learning tasks like feature extraction, feature selection or dimensionality
reduction are also provided as estimators.

The constructor of an estimator object takes as input a set of named constant
hyper-parameters. At initialization, all given parameters are attached to the
estimator instance as public attributes\footnote{Python does not actually
distinguish between public and private  attributes or methods. By convention
however, any identifier prefixed with an underscore is considered a non-public
part of the API~\citep{pythontut}.} and serve to determine the estimator
behavior (e.g., the $C$ constant in SVMs). Parameters set as public attributes
are also used for introspection purposes, e.g., in model selection routines. To
ease their use, appropriate default hyper-parameter values are provided for all
built-in estimators. These default values are set to be relevant in most common
situations in order to make estimators as efficient as possible \textit{out-of-box}
for non-experts. Finally, the constructor of an estimator never performs
any actual learning, nor does it even see any training dataset. Its only goal is
to set the parameter values and freeze the behavior of the estimator.

The estimator interface also defines a \texttt{fit} method. Invoking this method
causes the estimator to learn from a training set (e.g., supplied as two arrays
\texttt{X} and \texttt{y} in supervised learning estimators). Its task, from an
API point of view, is to run a learning algorithm and to determine model-
specific  parameters from the training data and set these as attributes on the
estimator object. As a convention, the parameters learned by an estimator are
exposed as public attributes\footnote{Note that the learned parameters are
different from the hyper-parameters given to the constructor. To distinguish
between the two, learned parameters are all suffixed with a trailing underscore
(e.g., \texttt{clf.coef\_} in the case of linear models is the vector of
coefficients learned from the data).} on the trained object.

% Importantly, the parameters learned by an estimator
% are exposed as \textit{public} attributes on the trained object.
% This facilitates model inspection
% and makes it possible to train a model using scikit-learn,
% export it to a (standardized or custom) external data format
% and use the learned parameters in a different piece of code,
% perhaps a prediction algorithm implemented in a different language
% or one that combines the model parameters with other information
% to build a custom model.
% % XXX Hard to follow phrasing below
% Of course, it also forces the developers to think carefully
% about what might, in a more ``black box''-style toolkit,
% be considered the internals of objects,
% since authors of client code are allowed to rely
% on the names and formats of model parameters.

As an example, let us consider a supervised learning task using linear regression.
Given the API defined above, solving this problem is as simple as:
\begin{center}
\texttt{clf = LinearRegression(penalty="l1").fit(X, y)}
\end{center}
In this snippet, a \texttt{LinearRegression} estimator is first initialized by
setting the \texttt{penalty} hyper-parameter to \texttt{l1}, hence enabling $L1$
regularization, and using default values for all other hyper-parameters. Upon
\texttt{fit}, the model is then learned from the arrays \texttt{X} and
\texttt{y}. Also, since all estimators share the same interface, using a
different learning algorithm is as simple as replacing the constructor. For
example, to build a Random Forest on the same training data, one would simply
have to replace \texttt{LinearRegression(penalty="l1")} in the snippet above by
\texttt{RandomForestClassifier()}. Let us also note that the snippet relies on the
fact that the \texttt{fit} method always returns the object it was called
on\footnote{I.e., the \texttt{self} argument it was called on.}, hence making
method chaining  possible.

In \textit{scikit-learn}, classical learning algorithms are not the only objects
to be implemented as estimators. For example, preprocessing routines (e.g.,
vectorization of text documents, scaling of attributes, etc) or feature
extraction techniques also implement the same interface. Even those estimators
that are entirely stateless and therefore do not require the \texttt{fit} method
to perform any useful work implement the estimator interface. As we will
illustrate in the next sections, this design pattern is indeed of prime
importance for consistency, composition and parameter validation reasons.

\textbf{Predictors.} The \textit{predictor} interface defines a \texttt{predict}
method taking as input some dataset \texttt{X} and producing as output
predictions for \texttt{X} based on the learned parameters of the estimator. In
the case of supervised learning estimators, this method typically returns the
predicted labels or values computed by the model. Some unsupervised learning
estimators may however also implement that interface: for example, in clustering
estimators, \texttt{predict} returns the identifiers of the closest clusters for
the samples in \texttt{X}.

Along with the \texttt{predict} method, predictors may also implement methods
for quantifying the certainty of a prediction. In the case of estimators
implementing linear models, the \texttt{decision\_function} method can be used
to measure the distance of the samples from the discriminating hyperplane. Some
predictors also provide a \texttt{predict\_proba} method which returns
class probabilities for the input samples \texttt{X}.

Finally, predictors should also provide a \texttt{score} function to assess the
goodness of fit on some input data. In supervised learning estimators, this
method takes as inputs a dataset \texttt{X} and \texttt{y} and typically
computes the coefficient of determination between \texttt{y} and
\texttt{predict(X)} in the case of classification or the mean squared error in
the case of regression. Not however that, from an API point of view, the only
guideline is that the \texttt{score} method returns a numerical value such that
the higher the better. As such, some unsupervised learning estimators also
exposes a \texttt{score} function, computing for instance the likelihood of the
given data.

Continuing with the previous example, predictions on unseen data
\texttt{X\_test}, \texttt{y\_test} can therefore be obtained and evaluated using
the following snippet:
\begin{center}
\texttt{y\_hat = clf.predict(X\_test)}\\
\texttt{score = clf.score(X\_test, y\_test)}
\end{center}

\textbf{Transformers.} Since it is common to modify or filter data before
feeding it to a learning algorithm, some estimators in the library implement a
\textit{transformer} interface which defines a \texttt{transform} method. It
takes as input some dataset \texttt{X} and yields as output a transformed
version of \texttt{X}\footnote{From an implementation point of view, it may
sometimes be more efficient to implement \texttt{fit} and \texttt{transform} in
a single method. In that case, transformers may expose an additional
\texttt{fit\_transform} method.}. In particular, all preprocessing algorithms
and feature selection routines are offered as transformers within the library.

In our example, if one wanted to transform the inputs \texttt{X} before
fitting the linear regression estimator, e.g., by standardizing the feature values,
one could thus simply write:
\begin{center}
\texttt{X = StandardScaler().fit(X).transform(X)}
\end{center}

% The bulk of the scikit-learn application programming interface
% consists of so-called \textit{estimator} objects.
% All learning algorithms for classification, regression and clustering
% are offered as such objects,
% but so are feature extraction, feature selection and dimensionality reduction
% algorithms.

% Construction and application of an estimator object
% proceeds in a way that is somewhat reminiscent of partial function application.
% The constructor, \texttt{\_\_init\_\_}\footnote{
%   Strictly speaking, \texttt{\_\_init\_\_} is Python's
%   class \textit{initialization} method,
%   while \texttt{\_\_new\_\_} is the constructor.
%   The distinction is not relevant to the present description of scikit-learn.}
% takes a set of named (hyper-)parameters
% and attaches these to the estimator as public attributes.\footnote{
%   The Python language does not actually distinguish between public and private
%   attributes and methods,
%   but by convention, any identifier prefixed with an underscore (\texttt{\_})
%   is considered ``a non-public part of the API\ldots
%   an implementation detail and subject to change without notice.''
%   \Citep{pythontut}
%   }
% It never performs actual learning, nor does it even see a training set.

% The algorithmic workhorse of most estimators, and one of the few methods
% that all estimators have in common, is called \texttt{fit}.
% Invoking this method causes the estimator to learn from a training set,
% supplied as one or more arguments.
% Its task, from an API point of view,
% is to determine (via a learning algorithm) model-specific parameters
% from a training set and set these as attributes on the estimator object,
% which can then be used to transform data or make predictions.
% As in other object-oriented machine learning libraries
% (e.g.\ Weka's Java API)
% no distinction is made in the type hierarchy
% between what is variously called an estimator or a \textit{learner}
% in the machine learning literature,
% and the \textit{model} that results from applying an estimator to data.
% When the \texttt{fit} method has been called,
% an estimator object serves dual purpose as a model
% (and is referred to as such in the documentation).

% An example to clarify the partial application style
% is the following code snippet.
% If $X$ is a dataset in the format described below and $k$ is some integer,
% then the Python code to train a $k$-means clustering model on it
% can be written as
% % XXX we could use Pygments for syntax highlighting
% \begin{verbatim}
% km = KMeans(n_clusters=k).fit(X)
% \end{verbatim}
% This snippet relies on the fact that the \texttt{fit} method
% always returns the object it was called on (its ``\texttt{self}''),
% making a method chaining style possible.

% A \texttt{fit} method is present even on objects
% that perform seemingly mundane tasks such as vectorizing text documents
% for subsequent learning.
% Such classes, however, can be said to learn their vocabulary
% from the training corpus,
% and follow the conventions set out above.
% Even those estimators that are entirely stateless and therefore
% do not require a \texttt{fit} method to perform useful work,
% nevertheless have such a method for consistency and composibility,
% and to perform parameter validation.
% Examples of this latter kind of objects are the normalizing transformer,
% kernel approximation transformers
% \citep{rahimi2007random, li2010random, vedaldi2010efficient},
% % XXX is it relevant that they are exactly two? Propose s/two/the
% and two feature extraction classes that implement the ``hashing trick''
% \citep{weinberger2009}.


\subsection{Composition}

A distinguishing feature of the \textit{scikit-learn} API is its ability to
compose new estimators from several base estimators. Composition mechanisms can
be used to combine typical machine learning workflows into a single object which
is itself an estimator. Through duck typing, composed
estimators can be used wherever usual estimators can be used. In particular,
they integrate within model selection routines, allowing one to optimize at once
every step of a complex workflow. Composition of estimators can be done in two
ways: either sequentially through \texttt{Pipeline} objects or in a parallel
fashion through \texttt{FeatureUnion} objects.

\texttt{Pipeline} objects are used for chaining multiple estimators into a
single one. This is useful since a machine learning workflow typically involves
a fixed sequence of steps in processing data (e.g., feature extraction,
centering / scaling / normalization, feature selection, learning and making
predictions). For convenience reasons, those steps can be combined into a
pipeline such that \texttt{fit} and \texttt{predict} need to be called only once
to fit the whole sequence. A pipeline also allows for the hyper-parameters of
all the base estimators to be grid-searched all at once. From an API point of
view, a \texttt{Pipeline} object takes as input a sequence of $N$ estimators,
where all $N-1$ first are transformers and the last is either a predictor or a
transformer. Calling \texttt{fit} on the pipeline is the same as calling
\texttt{fit} on each estimator in turn, \texttt{transform} the input and then
pass it on to the next step. The pipeline exposes all the methods the last
estimator in the pipe exposes. That is, if the last estimator is a predictor,
the pipeline can be used itself as a predictor. If the last estimator is a
transformer, then the pipeline is itself a transformer. \textcolor{red}{TODO:
Talk about nested attribute syntax.}

A \texttt{FeatureUnion} object is a transformer combining multiple transformers
into a single one that joins their outputs. From an API point of view, a
\texttt{FeatureUnion} takes as input a list of transformers. Calling
\texttt{fit} on the union is the same as calling \texttt{fit} independently on
each of the transformers. For transforming data, the base transformers are
applied in parallel are their outputs are concatenated end-to-end into a larger
output.

As an example, the following snippet illustrates how to create a complex worflow
computing both PCA and Kernel PCA components on \texttt{X\_train} (through a
\texttt{FeatureUninon}), selecting the best of those features and then feeding
them to a linear model:
\begin{verbatim}
Pipeline([("pca", FeatureUnion([('linear_pca', PCA()),
                                ('kernel_pca', KernelPCA())])),
          ("feature_selection", SelectKBest()),
          ("linear_model", LinearRegression())
]).fit(X_train, y_train).predict(X_test)
\end{verbatim}

As the above snippet shows, \texttt{Pipeline} and \texttt{FeatureUninon}
can be combined to create complex objects. In particular, since both
are themselves estimators, they can be used to form nested workflows.

% A distinguishing feature of \textit{scikit-learn} estimators
% is that they can be composed into \texttt{Pipeline} objects.
% Such objects combine typical machine learning workflows
% (extracting features, centering/scaling/normalization,
% feature selection, making predictions)
% in a single object that is itself an estimator,
% and whose methods cause data to flow through the pipeline.
% More importantly, though, they can be used for
% systematic, automated model selection.
% This is best explained by example.

% By composing, say, a \texttt{CountVectorizer} object
% (which extracts term frequency features from text documents
% according to a bag-words model),
% a \textsf{tf--idf} transformer, a $\chi^2$ feature selector and a linear SVM,
% one obtains a pipeline object with various hyperparameters:
% whether to do stopword filtering,
% linear vs.\ logarithmic \textsf{tf}, number of features to keep,
% SVM regularization ($L_1$ or $L_2$ norm, strength $C$) a.o.
% Such a composite \texttt{Pipeline} object
% exposes all of these parameters using a special syntax;
% each estimators must be given a name,
% and if the SVM's name is, e.g., \texttt{linearsvm},
% then its $C$ parameter is exposed as \texttt{linearsvm\_\_C}.
% The pipeline's \texttt{set\_params} splits the parameter names
% on double underscores, uses the first part of the result
% to look up the estimator,
% and passes the second part to the estimator's own \texttt{set\_params}
% as a keyword argument; due to Python's dynamic parameter passing,
% the result is the same as if the user had called \texttt{set\_params}
% on the SVM themselves.
% (The double underscore syntax is not special to pipelines;
% it is also employed in ensemble estimators such as random forests
% % Is the following clear? Suggestions to replace "direct attributes" with"?
% It is chosen as a convention to avoid ambiguity, as direct
% attributes and methods of scikit-learn objects never contain
% double underscores.)

\subsection{Model selection}

\textcolor{red}{TODO: rewrite + describe Scorer API}

The aforementioned \texttt{Pipeline} object
can be passed as an argument to a model selection algorithm.
Available algorithms at the time of writing include random search
\citep{bergstra2012} and ``grid search''.
The latter, which we shall use in this example,
performs an exhaustive search through a grid formed by the Cartesian product
of a set of possible values for each parameter, specified by the user.
In our example, this grid might be given as:

\begin{align*}
         & \textsf{stopwords} \in \{0, 1\}                      \\
  \times & \; \textsf{tf} \in \{\textsf{linear}, \textsf{log}\} \\
  \times & \; n_\textsf{features} \in \{1000, 2000, 5000\}      \\
  \times & \; \textsf{norm} \in \{L_1, L_2\}                    \\
  \times & \; C \in \{10, 100, 1000\}
\end{align*}

At each point in this grid, $k$-fold cross validation is run
to estimate the performance of the estimator according to some measure
(e.g.\ accuracy, $F_\beta$-score or a host of clustering quality metrics)
and the best performing set of hyperparameters is stored
on the \texttt{GridSearchCV} object, again, as a public attribute.

\section{Implementation}

scikit-learn is primarily implemented in Python and Cython
\citep{behnel2011cython},
a language that extends Python with static typing
and a compiler that produces C extension modules
for the Python runtime system.
In addition, it includes (modified versions of)
the \textsf{LIBSVM} and \textsf{LIBLINEAR} libraries
used for training support vector machines
and logistic regression models \citep{chang2011libsvm, fan2008}.
These libraries are both written in C{}\verb!++!
and wrapped using Cython modules.

As much as possible is implemented in ``pure'' Python,
to avoid the compile-edit-debug cycle needed for Cython
and to keep the codebase readable for as large an audience as possible.
However, since the speed of the current Python implementation
is not sufficient for numeric programming
and NumPy's vector operations are not appropriate for all use cases
(among other factors, because they often have linear space requirements),
some of the core algorithms have to be implemented in Cython.\footnote{
  The only Python implementation currently supported by scikit-learn
  is the reference implementation, CPython.
  The alternative Python implementation PyPy \citep{bolz2009tracing}
  promises to alleviate this problem by JIT compilation,
  but does not yet interoperate with NumPy and SciPy.
}
Examples include the aforementioned $k$-means implementation,
stochastic gradient descent for linear models
and some graph-based clustering algorithms.

While object-oriented techniques and design patterns
are used throughout scikit-learn,
they mostly serve to facilitate code reuse and are not considered a project goal.
Where possible, the Python principle of ``duck typing'' is exploited
to simplify the implementation and skip the introduction of superfluous classes.
This allows for extensibility and flexibility at the same time:
user-defined estimators that follow scikit-learn conventions
should be usable in pipelines and other composite objects
without them actually inheriting from scikit-learn base classes.

\section{Weaknesses of the NumPy/SciPy environment}

% This needs an introductory paragraph, but the best I could come up with was:
As can be expected, the NumPy/SciPy environment is not completely flawless.

% The following hedge needs to be in here; we don't want to diss our peers
It should be noted that this section is meant
as a summary of challenges faced by scikit-learn development
and an attempt at constructive criticism.
There is considerable overlap between the communities
that develop or contribute to NumPy, SciPy, scikit-learn, and Cython,
so the issues raised here may eventually be resolved
by the scikit-learn developers themselves.

\subsection{Sparse matrix support}

SciPy's handling of sparse matrices is one source of confusion
for scikit-learn developers and users alike.
The \texttt{scipy.sparse} package offers quite full-featured support
for this kind of data structure,
which is crucial to achieving performance and scalability
in the face of high-dimensional inputs
(including, but not limited to, NLP applications).
However, the interface exposed by its classes
deviates from the NumPy array interface, sometimes subtly so,
sometimes entirely.  % XXX is this grammatically correct?
For example, while NumPy arrays have a variable number of dimensions,
so that vectors can be considered 1-d arrays
(without a distinction between row and column vectors),
matrices are 2-d, etc., \texttt{scipy.sparse} matrices are always 2-dimensional,
meaning that special care has to be taken when handling sparse vectors.

The development team has built up sufficient familiarity
with the sparse/array distinction to factor out common patterns
for supporting both sparse and dense input data structures in many estimators.
In other cases, though, sparse matrix support has to implemented
on a per-estimator basis.
This situation is aggravated by the lack of a C API for sparse matrices,
so that Cython modules have to implement their own version
of operations like sparse dot products
that ideally would have been handled by a library.

However, users of scikit-learn cannot always be expected
to have similar familiarity, and in some cases
(e.g., when users want to leverage feature extraction code
outside the context of a \texttt{Pipeline})
they will have to face the \texttt{scipy.sparse} interface.
This is, of course, as much an artifact of the design choice
to not implement separate classes and interfaces for samples/instances
as it a consequence of SciPy's implementation of sparse matrices.

\subsection{Parallelism}

At this point in time, support for multi-core programming in Python,
at least in the CPython reference implementation targeted by scikit-learn,
is downright weak.
The interpreter supports multithreading, but due to the ``Global Interpreter Lock'' (GIL),
no two threads in a process may at the same time execute Python code.
It follows that multithreading is not a viable solution
even two embarrassingly parallel problems,
such as fitting and evaluation multiple models in a grid search procedure.
The workaround is to use Python's multiprocessing module,
which emulates threads using multiple operating system-level processes,
but this in turn leads to high memory use and overhead
because of the copying of training and test sets
into the various processes' address spaces.\footnote{
  The fact that copying occurs at all is an artifact of NumPy's implementation;
  no attempt at implementing a multiprocessing module
  that shares NumPy arrays using copy-on-write semantics or memory mapping
  has so far been completed.}

\subsection{Python/Cython mismatch}

The Python/Cython divide is another source of potential difficulties
both for users and developers/contributors.
While Cython makes writing performance-critical code possible
in a language that is very much like Python,
doing so properly still requires some knowledge of C.
One source of trouble has been the various fixed-size integer types
that are used by the Python and NumPy C runtimes,
whose sizes are practically all platform-dependent.
Choosing the appropriate integer type from among
\texttt{int}, \texttt{size\_t}, \texttt{Py\_ssize\_t} and \texttt{npy\_intp}
can require knowledge of the Python extension API,
the NumPy C API and the C standard.
Neither Cython nor C compilers provide much in the way of static checking
for the right type,
while picking the wrong type may cause segmentation faults
when users try to use arrays that are too large for the type used
and using a dynamic type often leads to unacceptable slowdowns.

Another problem concerns the mismatch between on the one hand Python functions,
which are first-class runtime objects endowed with methods and metadata
that identify them as ``callable'',
that dynamically check their number and types of arguments,
and that can return any Python type of object,
and compiled C or Cython functions on the other hand,
which exhibit static typing.
Where it may sometimes be convenient for estimators
to take functions as arguments
(e.g., in the case of kernel machine learning,
where one might like to use a custom kernel),
but such usage is not at present possible and may never be.
Passing C or Cython-compiled functions would require an API for the kernel method
to be implemented at the C or Cython level,
only to be used from compiled code.
Passing arbitrary Python functions would be possible,
but performance can be expected to degrade significantly
because of the runtime checks needed for each invocation of the kernel function.
(The current solution in scikit-learn is to let the user pass not a kernel function,
but a precomputed Gram matrix for an entire set of objects.
The linear, polynomial and RBF kernels are treated as special cases in
the \textsf{LIBSVM}-based SVM learner, where they are implemented in C{}\verb!++!.)

\section{Comparison with other packages}

ToDo.

\section{Future work}

ToDo.

\bibliographystyle{plainnat}
\DeclareRobustCommand{\VAN}[3]{#3}
\bibliography{paper}

\end{document}
