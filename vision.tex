\newcommand{\Xtr}{\texttt{X\_tr}}
\newcommand{\Xte}{\texttt{X\_te}}
\newcommand{\ytr}{\texttt{y\_tr}}
\newcommand{\yte}{\texttt{y\_te}}

\section{Project vision}

\subsection{General principles}

ToDo.

\subsection{Data formats}

As a machine learning library, one of the most important decisions we had to
make is how to represent data.  Rather than reinventing the wheel, we opted for
NumPy multidimensional arrays \Citep{vanderwalt2011} for dense data and SciPy
sparse matrices for sparse data.  While those may seem like a ``bare''
representation of data when compared to more object-oriented representations
(Weka is an example of this style of representation), it brings the prime
advantage of allowing us to rely on NumPy and SciPy's \textit{vector
operations} which are often orders of magnitude faster than the corresponding
Python loops, while at the same time keeping the code short and readable.
Conversion to NumPy and SciPy formats is usually easy and many scientific users
of Python will already be used to such formats, since they are pervasive in
other scientific Python packages.  For tasks where input is likely to consist
of text files or semi-structured objects, we provide ``vectorizers'' -- objects
that efficiently convert such data to the NumPy or SciPy formats.
% FIXME: elaborate what is meant below
%Finally, offering a custom datatype for samples would require a conversion
%\textit{in any case}.

In the remaing of the paper, we use the following notation. Training data are
denoted by \Xtr ~and test data (i.e., unseen data to which the learning
algorithm need be able to generalize) by \Xte.  Following the discussion above,
we represent \Xtr ~and \Xte ~by 2-dimensional NumPy arrays or SciPy sparse
matrices. Using NumPy/SciPy's notation, \Xtr\texttt{[i, j]} denotes the
$i^{th}$ training sample's $j^{th}$ feature. For supervised learning tasks,
such as classification or regression, we denote the training and test labels by
\ytr ~and \yte, respectively. We typically store \ytr ~and \yte ~as
1-dimensional NumPy arrays (\ytr\texttt{[i]} denotes the $i^{th}$ training
sample's label value).
